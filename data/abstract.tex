% 中英文摘要和关键字
\begin{cabstract}

    骨科相关感染(假体关节感染和骨折相关感染)会导致慢性疼痛和其他造成患者不适的症状,对人类的身心健康危害的同时还有高昂的医疗费用。因此,对患者的及时评估与准确诊断是正确治疗、预防并发症和良好预后的关键。核医学影像是骨科相关感染中常见的诊断手段之一。动态骨显像用于诊断假体关节感染,PET/CT则用于诊断骨折相关感染。医生在动态骨显像中观察每张图像中的组织摄取形态和不同图像之间时序上的生理代谢差异这些影像学特征作为依据来进行评估与诊断。在PET/CT中则通过观察软组织的损伤程度、骨骼的断裂程度、以及高摄取区域呈现的形式和程度等多方面的影像学特征,进行综合分析并作出最终的诊断决策。

    随着人工智能的发展,越来越多计算机视觉的深度学习方法以其优异的性能、较高的准确率和较短的推理时间的特点被逐步应用于医学影像之中。然而,对于骨科相关感染,目前还没有基于深度学习的全自动诊断系统的相关研究。采用深度学习方法来实现该系统时,需要去研究解决一系列问题。例如数据集的缺乏、动态骨显像存在生理高摄取区域的干扰、具有时序性的动态骨显像包含三种相、PET/CT中病灶形态和位置多样且只占据整个影像中一小部分以及PET和CT均为三维影像等等。因此,鉴于核医学影像及相关疾病的特征,可以将骨科相关感染的自动诊断系统视为医学影像领域中的检测与分类任务。针对上述问题,本文的主要研究内容总结如下:

    (1)\textbf{数据集的构建与预处理方法的设计:}对于骨科相关感染数据集的缺乏,本文作者与上海市第六人民医院核医学科合作,共同构建了两个关键数据集。其中,动态骨显像数据集纳入了从2016年1月至2021年6月之间455名患者的影像数据和相应的完整诊断报告,而PET/CT影像数据集纳入了2016年11月到2021年12月之间281名患者的相关数据。在医学专家的指导下,对数据集进行了详细的标注工作,以确保数据的准确性和可靠性。其次,针对动态骨显像和PET/CT影像,分别设计了一种有效的数据预处理方法,用于优化图像质量、标准化数据格式,以及降低无关区域和生理高摄取区域对模型性能的不利影响。

    (2)\textbf{基于\(^{99m}\)Tc-MDP动态骨显像的假体关节感染辅助诊断框架:}针对动态骨显像中假体关节感染诊断任务,设计了结合三维卷积和ConvLSTM的辅助诊断分类模型DBS-eNet,用以更好地利用动态骨显像中不同相以及相中的时序性图像序列。三维卷积在时序上提取不同相下每一张图像的生理摄取形态特征。ConvLSTM捕捉和丰富单一图像生理形态摄取特征的同时还捕获图像序列中生理代谢的差异变化特征。实验结果表明,对于假体膝关节感染和假体髋关节感染的诊断任务,该框架在五折交叉验证中优于其他卷积神经网络,准确率分别达到了86.48\%和86.33\%;在独立验证中与核医学专家们比较,准确率在膝关节上提升了13.55\%,髋关节上保持一致。

    (3)\textbf{基于\(^{18}\)F-FDG PET/CT三维影像的下肢骨折相关感染检测与诊断框架:}针对PET/CT影像中下肢骨折相关感染诊断任务,提出了由病灶检测网络和病灶诊断分类网络构成的两阶段框架3DFRINet。病灶检测网络通过双分支设计和注意力模块可以结合两个模态的多尺度细节特征,从而准确地检测出病灶区域。病灶诊断分类网络通过最大强度投影将立体影像转换为平面图像,获取有效特征的同时对候选区域进一步挖掘病灶的代谢形态特征,实现对骨折相关感染的准确分类。实验结果表明,该框架在下肢PET/CT影像中病灶检测性能优于其他YOLO模型,AP\(_{25}\)达到了0.9234;在骨折相关感染病灶的诊断分类中,相比较于初级核医学医生准确率提升了11.27\%,比高级核医学医生则提升了4.23\%。

    (4)\textbf{骨科相关感染的自动诊断与可视化方法:}为了构建一个假体关节感染和骨折相关感染的自动化辅助诊断系统,本文采用了PyQt6框架,设计了一个骨科相关感染的自动诊断与可视化方法,用以提供全面而便捷的可视化浏览和分析功能。该方法整合了(2)和(3)的模型,支持假体关节感染的诊断分类和骨折相关感染的病灶自动检测和分类。此外,该方法支持二维、三维和PET/CT融合可视化,以满足不同医学影像模态的需求。该方法在可视化方面提供多种辅助功能,二维中包括坐标与医学影像值的显示、切换视图等;三维中包括旋转、缩放等。

\end{cabstract}

\ckeywords{医学影像,骨科相关感染,图像分类,目标检测,可视化}

\begin{eabstract}
    ...
\end{eabstract}

\ekeywords{Medical imaging, Orthopaedic Related infection, Image classification, Object detection, Visualization}
