% 中英文摘要和关键字
\begin{cabstract}

    骨科相关感染(假体关节感染和骨折相关感染)会导致慢性疼痛和其他造成患者不适的症状,对人类的身心健康危害的同时还需要高昂的医疗费用。因此,对患者的及时评估与准确诊断是治疗、预防并发症和良好预后的关键。核医学影像是骨科相关感染中常见的诊断手段之一。动态骨显像用于诊断假体关节感染,PET/CT则用于诊断骨折相关感染。医生在动态骨显像中观察每张图像中的组织摄取形态和不同图像之间时序上的生理代谢差异这些影像学特征作为依据来进行评估与诊断。在PET/CT中则通过观察软组织的损伤程度、骨骼的断裂程度、以及高摄取区域呈现的形式等多方面的影像学特征,进行综合分析并作出最终的诊断决策。

    随着人工智能的发展,越来越多计算机视觉的深度学习方法以其优异的性能、较高的准确率和较短的推理时间等特点被逐步应用于医学影像之中。然而,对于骨科相关感染,目前还没有基于深度学习的全自动诊断方法的相关研究。采用深度学习来实现该方法时,需要去研究解决一系列问题。例如数据集的缺乏、动态骨显像存在生理高摄取区域的干扰、具有时序性的动态骨显像包含三种相、PET/CT中病灶形态和位置多样且只占据整个影像中一小部分以及PET和CT均为三维影像等等。因此,鉴于核医学影像及相关疾病的特征,可以将骨科相关感染的自动诊断方法视为医学影像的计算机辅助诊断领域中的检测与分类任务。针对上述问题,本文的主要研究内容总结如下:

    (1)数据集的构建与预处理方法的设计:对于骨科相关感染数据集的缺乏,本文作者与上海市第六人民医院核医学科合作,共同构建了两个关键数据集。其中,动态骨显像数据集纳入了从2016年1月至2021年6月之间455名患者的影像数据和相应的完整诊断报告,而PET/CT影像数据集纳入了2016年11月到2021年12月之间281名患者的相关数据。在医学专家的指导下,对数据集进行了详细的标注工作,以确保数据的准确性和可靠性。其次,针对动态骨显像和PET/CT影像,分别设计了有效的数据预处理方法,用于优化图像质量、标准化数据格式,以及降低无关区域和生理高摄取区域对模型性能的不利影响。

    (2)基于\(^{99m}\)Tc-MDP动态骨显像的假体关节感染辅助诊断框架:针对动态骨显像中假体关节感染诊断任务,构建了一种辅助诊断框架,由数据预处理和分类模型DBS-eNet组成。DBS-eNet结合了三维卷积和ConvLSTM,可以更好地利用动态骨显像中不同相以及相中的时序性图像序列。其中,三维卷积在时序上提取不同相下每一张图像的生理摄取形态特征,而ConvLSTM捕捉和丰富单一图像生理形态摄取特征的同时还捕获图像序列中生理代谢的差异变化特征。实验结果表明,对于假体膝关节感染和假体髋关节感染的诊断任务,该框架在五折交叉验证中优于其他卷积神经网络,准确率分别达到了86.48\%和86.33\%;在独立验证中与核医学专家们比较,准确率在膝关节上提升了13.55\%,髋关节上保持一致。

    (3)基于\(^{18}\)F-FDG PET/CT三维影像的下肢骨折相关感染检测与诊断框架:针对PET/CT影像中下肢骨折相关感染诊断任务,提出了由病灶检测网络和病灶诊断分类网络构成的两阶段框架3DFRINet。病灶检测网络通过双分支设计和注意力模块可以结合两个模态的多尺度细节特征,从而准确地检测出病灶区域。病灶诊断分类网络通过最大强度投影将立体影像转换为平面图像,获取有效特征的同时对候选区域进一步挖掘病灶的代谢形态特征,实现对骨折相关感染的准确分类。实验结果表明,该框架在下肢PET/CT影像中病灶检测性能优于其他YOLO模型,AP\(_{25}\)达到了0.9234。在骨折相关感染病灶的诊断分类中,相比较于初级核医学医生准确率提升了11.27\%,比高级核医学医生则提升了4.23\%。

    (4)骨科相关感染的自动诊断与可视化方法:为了完善假体关节感染和骨折相关感染的自动化辅助诊断流程,本文采用了PyQt6框架,设计了一个骨科相关感染的自动诊断与可视化方法,用以提供全面而便捷的可视化浏览和分析功能。该方法整合了(2)和(3)的框架,支持假体关节感染的辅助诊断和骨折相关感染的辅助检测诊断。此外,该方法支持二维、三维和PET/CT融合可视化,以满足不同医学影像模态的需求。该方法在可视化方面提供多种影像分析功能,包括坐标与影像值的展示、切换视图、旋转和缩放等等。

\end{cabstract}

\ckeywords{医学影像,骨科相关感染,图像分类,目标检测,可视化}

\begin{eabstract}

    Orthopedic related infection (prosthetic joint infection and fracture related infection) causes chronic pain and other symptoms that cause discomfort to the patient, posing a risk to human physical and mental health along with high medical costs. Therefore, timely evaluation and accurate diagnosis of patients are essential for treatment, prevention of complications and good prognosis. Nuclear medicine imaging is one of the common diagnostic tools in orthopedic related infections. Dynamic bone scintigraphy is used to diagnose prosthetic joint infection, while PET/CT is used to diagnose fracture related infection. In dynamic bone scintigraphy, the physician observes the tissue uptake pattern in each image and the physiological and metabolic differences in the time series between images as the basis for evaluation and diagnosis. In PET/CT, various imaging features such as the extent of soft tissue damage, the degree of fracture of the bone, and the form of the uptake area are observed and analyzed to make a final diagnostic decision.

    With the development of artificial intelligence, more and more deep learning methods for computer vision have been gradually applied to medical imaging with their excellent performance, high accuracy and short inference time. However, for orthopedic related infection, there is no research related to a fully automated diagnostic method based on deep learning. When deep learning is used to implement this method, a series of problems need to be researched and solved. For example, there is a lack of datasets, dynamic bone scintigraphy has interference from physiological high uptake regions, dynamic bone scintigraphy with temporal sequencing contains three phases, lesion morphology and location in PET/CT are diverse and occupy only a small portion of the whole image, and both PET and CT are three-dimensional images, and so on. Therefore, considering the characteristics of nuclear medicine images and related diseases, automatic diagnostic methods for orthopedic related infection can be regarded as a detection and classification task in the field of computer aided diagnosis of medical images. To address the above issues, the main research of this paper is summarized as follows:

    (1) Construction of datasets and design of the preprocessing method: For the lack of orthopedic related infection datasets, the author collaborated with the Nuclear Medicine Department of Shanghai Sixth People's Hospital to construct two key datasets. Among them, the dynamic bone scintigraphy dataset incorporated the imaging and corresponding complete diagnostic reports of 455 patients between January 2016 and June 2021, while the PET/CT imaging dataset incorporated the relevant data of 281 patients between November 2016 and December 2021. Detailed labelling work was carried out on the datasets under the guidance of medical experts to ensure the accuracy and reliability of the data. Next, effective data preprocessing methods were designed for dynamic bone scintigraphy and PET/CT images, respectively, for optimizing the image quality, standardizing the data format, and reducing the detrimental effects of irrelevant and physiologically high uptake regions on the model performance.

    (2) An assisted diagnostic framework for prosthetic joint infection based on \(^{99m}\)Tc-MDP dynamic bone scintigraphy: For the diagnostic task of prosthetic joint infection in dynamic bone scintigraphy, an assisted diagnostic framework is constructed, consisting of data preprocessing and classification model, DBS-eNet. DBS-eNet combines 3D convolution and ConvLSTM, which can better utilize temporal image sequences in different phases as well as in-phase from dynamic bone scintigraphy. Among them, 3D convolution extracts the physiological uptake morphological features of each image under different phases on the temporal sequence, while ConvLSTM captures and enriches the physiological morphological uptake features of a single image while also capturing the features of differential physiological metabolism changes in the image sequence. The experimental results show that for the diagnostic tasks of prosthetic knee infection and prosthetic hip infection, the framework outperforms other convolutional neural networks in five-fold cross-validation, with accuracies of 86.48\% and 86.33\%, respectively, and in independent validation comparing with nuclear medicine experts, the accuracies are improved by 13.55\% on the knee, and remain consistent on the hip.

    (3) A framework for the detection and diagnosis of lower extremity fracture related infection based on \(^{18}\)F-FDG PET/CT three-dimensional imaging: For the diagnosis task of lower extremity fracture related infection in PET/CT images, a two-stage framework 3DFRINet consisting of a lesion detection network and a lesion diagnostic classification network is proposed. The lesion detection network can accurately detect lesion regions by combining multi-scale detailed features of both modalities through the dual-branching design and the attention module. The lesion diagnostic classification network converts the cubic image into the planar image through maximum intensity projection, acquires effective features while further mining the metabolic-morphological features of the lesion for the candidate region, and achieves accurate classification of fracture related infection. The experimental results show that the framework outperforms other YOLO models in lesion detection in lower limb PET/CT images, with an AP\(_{25}\) of 0.9234. In the diagnostic classification of fracture related infection lesions, the accuracy of the framework is improved by 11.27\% in comparison with junior nuclear medicine physicians, and 4.23\% compared with senior nuclear medicine physicians.

    (4) The automated diagnostic and visualization method for orthopedic related infection: To improve the automated assisted diagnosis process for prosthetic joint infection and fracture related infection, this paper adopts the PyQt6 framework to design an automated diagnosis and visualization method for orthopedic related infection, which is used to provide comprehensive and convenient visualization browsing and analysis functions. The method integrates the frameworks of (2) and (3) to support assisted diagnosis of prosthetic joint infection and assisted detection and diagnosis of fracture related infection. In addition, the method supports 2D, 3D and PET/CT fusion visualization to meet the needs of different medical imaging modalities. The method provides a variety of image analysis functions in terms of visualization, including the display of coordinates and image values, switching views, rotation and scaling, etc.

\end{eabstract}

\ekeywords{Medical imaging, Orthopedic related infection, Image classification, Object detection, Visualization}
