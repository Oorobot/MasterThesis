\chapter{总结与展望}

\section{总结}

本文聚焦于在核医学影像中进行骨科相关感染疾病的辅助诊断,具体涉及假体关节感染和骨折相关感染。研究指出动态骨显像具有时间序列特性且包含三种不同阶段的相,PET和CT可以提供更全面的三维影像信息,以及动态骨显像和PET/CT中均存在自然高摄取区域的干扰等等。针对这些问题和特点,本研究提出了新颖的辅助分类诊断方法,包括基于动态骨显像的假体关节感染辅助分类诊断方法、基于下肢PET/CT影像中骨折相关感染的检测与诊断方法和骨科相关感染的自动诊断与可视化方法。本文的主要工作总结如下:

(1)面对骨科相关感染领域医学影像数据集的不足,本文作者与上海市第六人民医院核医学科进行了合作,共同构建了两个核心数据集。动态骨显像数据集涵盖了2016年1月至2021年6月期间455位患者的影像资料及其匹配的完备诊断报告。至于PET/CT数据集,则收集了2016年11月至2021年12月期间281位患者的医学影像和诊断报告。在医学专家的指导下,对这两个数据集进行了精确的数据标注,确保其高度的准确性和可信度。此外,本文还特别为动态骨显像和PET/CT影像设计了专门的数据预处理方法,旨在统一数据格式,提升影像质量,并减少无关区域和自然高摄取区域带来的不利影响。

(2)为提升动态骨显像在假体关节感染诊断中的应用效果,本研究创造性地提出了一种融合三维卷积与ConvLSTM的辅助诊断分类模型——DBS-eNet。该模型充分挖掘了动态骨显像多个相及其内部的图像序列信息。通过三维卷积技术,模型能够在时间维度上有效提取每幅图像的生理摄取形态特征,而ConvLSTM则在强化图像的形态特征的同时,能够捕捉图像序列中生理代谢活动的动态变化。通过实验验证,DBS-eNet模型在处理假体膝关节感染与假体髋关节感染诊断任务时表现出色,准确率在五折交叉验证测试中分别达到了86.48\%和86.33\%,并在与核医学专家的独立验证比较中,膝关节诊断准确率提高了13.55\%,髋关节诊断准确率与专家持平。

(3)面向下肢骨折相关感染的PET/CT影像诊断任务,本研究提出了一个创新的两阶段框架——3DFRINet。该框架整合了专门设计的病灶检测网络与病灶诊断分类网络。病灶检测网络通过双分支网络的架构设计和注意力机制,高效融合不同模态下的多尺度特征,实现病灶的精确定位。紧随其后的病灶诊断分类网络,运用最大强度投影技术将三维影像信息投影至二维,不仅保留了关键特征,也深入分析了感染病灶的代谢形态,以达到精准分类的目标。实验结果显示,3DFRINet在下肢PET/CT影像病灶检测任务中表现卓越,超越了其他YOLO模型,其AP\(_{25}\)指标高达0.9234;而在对骨折相关感染病灶进行诊断分类时,相比于资历较浅的核医学医师,准确率提高了11.27\%,与资深核医学医师相比也提升了4.23\%。

(4)本研究旨在创建一个全自动的假体关节感染和骨折相关感染辅助诊断系统,为此,基于PyQt6框架开发了一套针对骨科相关感染的自动诊断与可视化方法。该方法融合了(2)和(3)的模型,不仅支持假体关节感染的诊断分类工作,还能自动检测并分类骨折相关感染的病灶。此外,为了适应各种医学影像模态,本方法支持二维、三维以及PET/CT融合影像的可视化分析。在功能性设计上,本方法为二维影像提供了诸如医学影像值及其坐标呈现、视图切换等功能;而在三维影像方面,则允许用户进行图像旋转、缩放等操作,以促进对复杂医学影像数据的理解和分析。

综上所述,本文实现了假体关节感染和骨折相关感染的完整诊断流程。实验对比、可视化方法以及案例分析都证实了本文所提方法的可信度。本文所提方法可以为医疗诊断提供了客观的建议,助力医学影像智能诊断领域向规范化及标准化迈进。

\section{展望}

本研究成果虽已取得一定的成就,但其仍不免有所限制,并且未来的研究仍存在发展方向:

(1)尽管本文所构建的数据集对于实验而言足够,但它的样本规模较小、类别分布不均及数据来源单一均可能限制了本文提出方法的泛化能力,并有可能引入偏误。

(2)在动态骨显像中对假体关节感染的诊断任务中,尤其是涉及髋关节影像时,目前尚需依赖人工手动标记感兴趣区域,这一过程增加了工作量并可能引入人为误差。未来的研究中,开发一套完全自动化的诊断系统将是有价值的探索方向。

(3)在使用PET/CT进行骨折相关感染诊断的任务中,双阶段模型第一阶段的病灶检测准确率直接影响到最终的诊断效果。进一步提高病灶检测的精度可以有效提高整体框架的表现。另外,人体中普遍存在的对称性骨骼结构(如股骨、胫骨及腓骨等)。在对称两侧的对比中,可以使得一些异常特征突出得更加明显。因此,未来研究可以考虑引入对比学习(Contrastive Learning)方法,以便在骨折相关感染的诊断中进一步挖掘和利用这一特性。

(4)在动态骨显像中,第三相——延迟相对假体关节感染的深度学习辅助诊断贡献不大,但缺乏实验验证。SPECT也被用于假体关节感染的诊断中,利用深度学习对SPECT进行分析,可能具有研究价值。PET/CT中,衰减校正与否对PET在深度学习中诊断效果的影响值得深入研究,同时伪影消除技术去除CT中的金属伪影是否能提高深度学习模型的性能一样值得研究。